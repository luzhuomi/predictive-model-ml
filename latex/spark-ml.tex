\documentclass{beamer}
%% fink install texlive
\usetheme{Warsaw}

\usepackage{amsmath,amssymb}
\usepackage{stmaryrd}
\usepackage{graphicx}
%\usepackage{haskell}
\usepackage{code}
%\usepackage{proof}
\usepackage{theorem}
\usepackage{pstricks} 
\usepackage{listings}
\usepackage{pgf-pie} % from https://code.google.com/p/pgf-pie/



% Martin's macros
   

% groups

%%\newtheorem{lemma}{Lemma}
%%\newtheorem{definition}{Definition}
%%\newtheorem{theorem}{Theorem}
%%\newtheorem{corollary}{Corollary}
%%\newtheorem{remark}{Remark}
%%\newtheorem{conjecture}{Conjecture}
%%\newtheorem{example}{Example}

% arrays

\newcommand{\bi}{\begin{array}[t]{@{}l@{}}}
\newcommand{\ei}{\end{array}}
\newcommand{\ba}{\begin{array}}
\newcommand{\ea}{\end{array}}
\newcommand{\bda}{\[\ba}
\newcommand{\eda}{\ea\]}
\newcommand{\bp}{\begin{quote}\tt\begin{tabbing}}
\newcommand{\ep}{\end{tabbing}\end{quote}}
\newcommand{\set}[1]{\left\{
    \begin{array}{l}#1
    \end{array}
  \right\}}
\newcommand{\sset}[2]{\left\{~#1 \left|
      \begin{array}{l}#2\end{array}
    \right.     \right\}}


% symbols

\newcommand{\good}{good}
 
\newcommand{\err}{{\bf W}}
\newcommand{\VV}{{\cal V}} 
\newcommand{\GT}{{\cal T}_G}
\newcommand{\KK}{{\cal K}}
\newcommand{\TT}{{\cal T}}
\newcommand{\TC}{\mbox{\it TC}}
\newcommand{\UL}{\mbox{UL}}
\newcommand{\sUL}{\mbox{\scriptsize UL}}
\newcommand{\true}{\mbox{\sf true}}
\newcommand{\Int}{\mbox{\it Int}}
\newcommand{\Bool}{\mbox{\it Bool}}
\newcommand{\SF}{\mbox{$\cal S$}}
\newcommand{\tauvec}{\bar{\tau}}
\newcommand{\tvec}{\bar{t}}
\newcommand{\kvec}{\bar{k}}
\newcommand{\xvec}{\bar{x}}
\newcommand{\muvec}{\bar{\mu}}
\newcommand{\alphavec}{\bar{\alpha}}
\newcommand{\betavec}{\bar{\beta}}
\newcommand{\gammavec}{\bar{\gamma}}
\newcommand{\thetavec}{\bar{\theta}}
\newcommand{\deltavec}{\bar{\delta}}
\newcommand{\bvec}{\bar{b}}
\newcommand{\typo}{\Gamma}
\newcommand{\tenv}{\Gamma}
\newcommand{\typoinit}{\typo_0}
\newcommand{\static}{S}
\newcommand{\dynamic}{D}
\newcommand{\sstatic}{{\scriptstyle S}}
\newcommand{\sdynamic}{{\scriptstyle D}}
\newcommand{\BTA}{\mbox{BTA}}
\newcommand{\baseshape}{\bot}
\newcommand{\TCT}{P}
\newcommand{\Haskell}{H98}
\newcommand{\TCTinit}{P_o}

% figures, rules


\newcommand{\tlabel}[1]{\mbox{(#1)}}
\newcommand{\fig}[3]
        {\begin{figure*}[t]#3\
        \caption{\label{#1}#2}\ \hrulefill\ \end{figure*}}
\newcommand{\figurebox}[1]
        {\fbox{\begin{minipage}{\textwidth} #1 \end{minipage}}}
\newcommand{\boxfig}[3]
        {\begin{figure*}\figurebox{#3\caption{\label{#1}#2}}\end{figure*}}
\newcommand{\myirule}[2]{{\renewcommand{\arraystretch}{1.2}\ba{c} #1
                      \\ \hline #2 \ea}}

% relations, operators

\newcommand{\sat}{\mbox{\it sat}}
\newcommand{\entail}{\mbox{\it entail}}
\newcommand{\unambig}{\mbox{\it unambig}}
\newcommand{\complete}{\mbox{\it complete}}

\newcommand{\projexclusion}[1]{\mbox{$\bar{\exists}#1$}}
\newcommand{\funinst}{\mbox{\it inst}}
%%\renewcommand{\inst}{\, \vdash^{\scriptstyle i} \,}
%%\newcommand{\inst}{\, \vdash \,}
\newcommand{\ileq}{\preceq}
\newcommand{\ieq}{\simeq}
\newcommand{\topannot}[1]{|#1|}
\newcommand{\gendelta}{\mbox{\it gen}_{\delta}}
\newcommand{\genbeta}{\mbox{\it gen}_{\beta}}
\newcommand{\gen}{\mbox{\it gen}}
\newcommand{\normalize}{\mbox{\it normalize}}
\newcommand{\fail}{\mbox{\it fail}}
\newcommand{\fixpt}{\mbox{${\cal F}$}}
\newcommand{\testpa}{\mbox{${\cal T}$}}
\newcommand{\addpa}{\mbox{${\cal A}$}}
\newcommand{\matchpa}{\mbox{${\cal M}$}}
\newcommand{\refmatchpa}{\mbox{${\cal R}$}}
\newcommand{\wft}[1]{\mbox{\it wft}(#1)}
\newcommand{\tv}{\mbox{\it fv}}
\newcommand{\stv}{\mbox{\it semfv}}
\newcommand{\range}{\mbox{\it range}}
\newcommand{\lub}{\wedge}
\newcommand{\tyrel}[2]{(#1 \preceq #2)}
\newcommand{\eqty}[2]{#1=#2}
%%\newcommand{\ts}{\, \vdash \,}
\newcommand{\turns}{\, \vdash \,}
\newcommand{\tsinst}{\, \vdash^i \,}
\newcommand{\tsttv}{\, \vdash^{ttv} \,}
\newcommand{\tsUL}{\, \vdash_{\mbox{\scriptsize UL}} \,}
\newcommand{\trc}{\, \vdash_{\scriptstyle \id{inf}} \,}
\newcommand{\trcp}{\, \vdash_{\scriptstyle inf}' \,}
\newcommand{\genrel}{\, \vdash^{\scriptstyle gen} \,}
\newcommand{\asubty}[2]{(#1 \leq_a #2)}
\newcommand{\ssubty}[2]{(#1 \leq_s #2)}
\newcommand{\fsubty}[2]{(#1 \leq_f #2)}
\newcommand{\subty}[2]{(#1 \leq #2)}
\newcommand{\doubleb}[1]{[\![ #1 ]\!]}
\newcommand{\id}[1]{\mbox{\it #1}}


\newcommand{\mynote}[1]{$\spadesuit${\bf #1}$\clubsuit$}

% spacing

\newcommand{\sgap}{\quad}
\newcommand{\bgap}{\quad\quad}

% reserved words

\newcommand{\mathem}{\sf}
\newcommand{\IN}{\mbox{\mathem in}}
\newcommand{\LET}{\mbox{\mathem let}}
\newcommand{\MLET}{\mbox{\mathem mlet}}
\newcommand{\LETREC}{\mbox{\mathem letrec}}
\newcommand{\FIX}{\mbox{\mathem fix}}
\newcommand{\CASE}{\mbox{\mathem case}}
\newcommand{\TYCASE}{\mbox{\mathem typecase}}
\newcommand{\ELSE}{\mbox{\mathem else}}
\newcommand{\IF}{\mbox{\mathem if}}
\newcommand{\OF}{\mbox{\mathem of}}
\newcommand{\THEN}{\mbox{\mathem then}}
\newcommand{\INST}{\mbox{\mathem instance}}
\newcommand{\OVER}{\mbox{\mathem overload}}
\newcommand{\CLASS}{\mbox{\mathem class}}
\newcommand{\WHERE}{\mbox{\mathem where}}


%%% Local Variables: 
%%% mode: latex
%%% TeX-master: "bta"
%%% End: 


\newcommand{\beb}{\begin{exampleblock}}
\newcommand{\eeb}{\end{exampleblock}}


\newcommand{\rev}[1]{{#1}^{\mbox{\scriptsize r}}}
\newcommand{\lquo}{\backslash}
\newcommand{\rquo}{/}
\newcommand{\diff}{-}
\newcommand{\isect}{\cap}
\newcommand{\mymid}{~\redtxt{\mid}~}

\newcommand{\ignore}[1]{}

\newcommand{\magtxt}[1]{{\magenta #1}}
\newcommand{\redtxt}[1]{{\red #1}}
\newcommand{\bluetxt}[1]{{\blue #1}}
\newcommand{\greytxt}[1]{{\gray #1}}
\newcommand{\mmleq}{\leq}
\newcommand{\pow}{\^{}}
\newcommand{\venv}{\Delta}
\newcommand{\mleq}{\mbox{\tt leq}}
\newcommand{\mas}{\mbox{\tt as}}
\newcommand{\subt}{<\!:}

\newcommand{\Nturns}{\, \vdash_{\mbox{\tiny lnf}} \,}

\newenvironment{ttprog}{\begin{trivlist}\item \tt
        \begin{tabbing}}{\end{tabbing}\end{trivlist}}


\newenvironment{grammar}{%
         \begin{center} \small%
         $\begin{array}{rcll}
         }{%
         \end{array}$\end{center}\ignorespaces%
         }

\newcommand{\rem}[1]{}

\begin{document}

\title{Introduction to Spark Machine Learning} 
\author{ \white 
 Copyright \copyright~ 2015. All Rights Reserved
Kenny Zhuo Ming Lu 
}
\institute{
\normalsize 
Kenny Lu \\ 
School of Information Technology \\ Nanyang Polytechnic
}
\date{\today} 


%\bibliographystyle{plainnat}

\frame{\titlepage} 

%\frame{\frametitle{Table of contents}\tableofcontents} 


%-------------------------------------------------------------------
%-------------------------------------------------------------------
\begin{frame}
\frametitle{What is Spark Machine Learning?}
\begin{itemize}
\item A library of tools developed for machine learning
\item Built over the Apache Spark framework
\end{itemize}
\end{frame}


%-------------------------------------------------------------------
%-------------------------------------------------------------------
\begin{frame}
\frametitle{Why Spark Machine Learning?}

\begin{itemize}
 \item Scalable
 \item Compatibility (Java, R, Python, Scala)
 \item Data integration (HDFS, Cassendra, ...)
\end{itemize}
\end{frame}


%-------------------------------------------------------------------
%-------------------------------------------------------------------
\begin{frame}[fragile]
\frametitle{Spark Machine Learnng Data Types}
\begin{itemize}
\item RDD
\item Vector
\item Labeled Points
\item Matrix
\end{itemize}  
\end{frame}

%-------------------------------------------------------------------
%-------------------------------------------------------------------
\begin{frame}[fragile]
\frametitle{Resilient Distributed Dataset}

\begin{itemize} 
\item RDD is an abstraction over a collection of data set being distributed
and partitioned across a cluster of worker machines, mostly in memory.
\item Programmers are not required to manage or to coordinate that
distributed and partitioned. RDD is fault tolerant.  
\item RDDs are initialized and managed by the SparkContext.
\end{itemize}
\end{frame}


%-------------------------------------------------------------------
%-------------------------------------------------------------------
\begin{frame}[fragile]
\frametitle{Resilient Distributed Dataset}

\beb{scala}
\begin{code}
val sc = new SparkContext("local", "shell")	
// an RDD of doubles 
val seriesX:RDD[Double] = sc
  .textFile("data/basic/series1.txt")
  .map(_.toDouble)
\end{code}
\eeb
\end{frame}

%-------------------------------------------------------------------
%-------------------------------------------------------------------


\begin{frame}[fragile]
\frametitle{RDD transformations are pure}

Recall from the previous section. Let {\tt r} denotes an RDD,
\begin{itemize}
\item {\tt r.map(f)} and {\tt r.flatMap(f)} applies {\tt f} to
  elements in {\tt r}.
\item {\tt r.filter(f)} filters away elements {\tt x} in {\tt r} which {\tt
    f(x)} yields {\tt false}.
\item assuming {\tt r} is a collection of key-value pairs, {\tt r.reduceByKey(f)} will
  shuffle the pairs and group them by keys. The values grouped under the same key will be
  reduced by {\tt f}. Data locality is exploit when possible.
\end{itemize}

\end{frame}


%-------------------------------------------------------------------
%-------------------------------------------------------------------


\begin{frame}[fragile]
\frametitle{Localizing an RDD}

\beb{An example of RDD}
\begin{code}
val sc = new SparkContext("local", "shell")	
// an RDD of doubles 
val seriesX:RDD[Double] = sc
  .textFile("data/basic/series1.txt")
  .map(_.toDouble)
val arr:Array[Double] = seriesX.collect.toArray
\end{code}
\eeb

\end{frame}


%-------------------------------------------------------------------
%-------------------------------------------------------------------


\begin{frame}[fragile]
\frametitle{Vector}

Vectors are local data collection.
\begin{itemize}
\item Dense vector - similar to an array. All values need to be specified.
\beb{Dense Vector}
\begin{code}
// Create a dense vector (1.0, 0.0, 3.0).
val dv: Vector = Vectors.dense(1.0, 0.0, 3.0)
\end{code}
\eeb
\end{itemize}
\end{frame}

%-------------------------------------------------------------------
%-------------------------------------------------------------------


\begin{frame}[fragile]
\frametitle{Vector}

Vectors are local data collection.
\begin{itemize}
\item Sparse vector - programmers are required to specify the size of
  the vector  as well as the non-zero values.
\beb{Sparse Vector}
\begin{code}
// Create a sparse vector (1.0, 0.0, 3.0) 
// by specifying its indices and values 
// corresponding to nonzero entries.
val sv1: Vector = Vectors
  .sparse(3, Array(0, 2), Array(1.0, 3.0))
// Create a sparse vector (1.0, 0.0, 3.0) 
// by specifying its nonzero entries.
val sv2: Vector = Vectors
  .sparse(3, Seq((0, 1.0), (2, 3.0)))
\end{code}
\eeb  
\end{itemize}
\end{frame}


%-------------------------------------------------------------------
%-------------------------------------------------------------------


\begin{frame}[fragile]
\frametitle{Labeled Point}

Labeled points are vectors that with an assigned/labeled values. They
are commonly used as the training data in algorithms
such as logistic regressions and SVM. Labeled point is a labeled value
(normally 0 or 1) with a vector.
\beb{Labeled points}
\begin{code}
// Create a labeled point with a positive label 
// and a dense feature vector.
val pos = LabeledPoint(1.0, Vectors.dense(1.0, 0.0, 3.0))
// Create a labeled point with a negative label 
// and a sparse feature vector.
val neg = LabeledPoint(0.0, 
  Vectors.sparse(3, Array(0, 2), Array(1.0, 3.0)))
\end{code}
\eeb
\end{frame}



%-------------------------------------------------------------------
%-------------------------------------------------------------------


\begin{frame}[fragile]
\frametitle{Matrix}


Matrices can be seen as two diemsnional vectors. 
\begin{itemize}
 \item Local Matrix
 \item Distributed Matrix
\end{itemize}
\end{frame}

%-------------------------------------------------------------------
%-------------------------------------------------------------------


\begin{frame}[fragile]
\frametitle{Local Matrix}

Local matrices are similar to vector
\beb{Local Matrix}
\begin{code}
// a 3 (rows) by 2 (columns) matrix
val dm: Matrix = Matrices
  .dense(3, 2, Array(1.0, 3.0, 5.0, 2.0, 4.0, 6.0))
\end{code}
\eeb
\end{frame}

%-------------------------------------------------------------------
%-------------------------------------------------------------------


\begin{frame}[fragile]
\frametitle{Distributed Matrix}

Distributed matrices are practically RDD of local vectors. 
\begin{itemize}
\item RowMatrix
\item IndexedRowMatrix
\item CoordinateMatrix
\end{itemize}

\end{frame}


%-------------------------------------------------------------------
%-------------------------------------------------------------------


\begin{frame}[fragile]
\frametitle{RowMatrix}

Row-oriented distributed matrix
\beb{RowMatrix}
\begin{code}
val rows: RDD[Vector] = MLUtils.loadVectors(sc, 
  "data/basic/vector1.txt")
// Create a RowMatrix from an RDD[Vector].
val mat: RowMatrix = new RowMatrix(rows)
// Get its size.
val m = mat.numRows()
val n = mat.numCols()
\end{code}
\eeb

\end{frame}


%-------------------------------------------------------------------
%-------------------------------------------------------------------


\begin{frame}[fragile]
\frametitle{IndexedRowMatrix}

IndexedRowMatrix is an RowMatrix with indices (keys) to the rows.
\beb{IndexedRowMatrix}
\begin{code}
val indexedRows : RDD[IndexedRow] = rows
  .zipWithUniqueId().map( idv => IndexedRow(idv._2,idv._1) )
// Create an IndexedRowMatrix from an RDD[IndexedRow].
val irmat: IndexedRowMatrix = 
  new IndexedRowMatrix(indexedRows)
\end{code}
\eeb

\end{frame}


%-------------------------------------------------------------------
%-------------------------------------------------------------------


\begin{frame}[fragile]
\frametitle{CoordinateMatrix}

A CoordinateMatrix is a distributed matrix backed by an RDD of its entries. 
Each entry is a tuple of (i: Long, j: Long, value: Double), where i is the row index, 
j is the column index, and value is the entry
value. A CoordinateMatrix is a sparse matrix

\beb{CoordinateMatrix}
\begin{code}
val local_entries : List[MatrixEntry] = 
  (for ( i <- (1 to 10); j <- (2 to 5) ) 
  yield MatrixEntry(i, j, 0.0)).toList
val entries: RDD[MatrixEntry] = 
  sc.parallelize(local_entries, 2)
// Create a CoordinateMatrix from 
 // an RDD[MatrixEntry].
val cmat: CoordinateMatrix = new CoordinateMatrix(entries)
\end{code}
\eeb

\end{frame}

%-------------------------------------------------------------------
%-------------------------------------------------------------------
\begin{frame}[fragile]
\frametitle{Machine learning Models}
With above data types, we can build and use many interesting models.
Spark ML comes with many builtin models, 
\begin{itemize}
\item Classification
  \begin{itemize}
  \item Logistic Regression
  \item SVM
  \item NaiveBayse
  \end{itemize}
\item Clustering
  \begin{itemize}
     \item KMeans
     \item LDA
  \end{itemize}
\item Regression
   \begin{itemize}
     \item Linear Regression
   \end{itemize}
\item Other algorithms
   \begin{itemize}
     \item PCA
     \item NGram
     \item Word2Vec
   \end{itemize}

\end{itemize}
We will see some of these in the exercise
\end{frame}


%-------------------------------------------------------------------
%-------------------------------------------------------------------


\begin{frame}[fragile]
\frametitle{References}


\begin{itemize}
\item The examples are adapted from
  \url{https://spark.apache.org/docs/latest/}
\item More models of the library can be found 
 \url{https://spark.apache.org/docs/latest/api/scala/}
 \url{https://spark.apache.org/docs/latest/api/java/}

\end{itemize}


\end{frame}

%-------------------------------------------------------------------
%-------------------------------------------------------------------


\end{document}
